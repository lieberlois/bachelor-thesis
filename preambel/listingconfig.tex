% "define" Scala
\lstdefinelanguage{scala}{
  morekeywords={abstract,case,catch,class,def,%
    do,else,extends,false,final,finally,%
    for,if,implicit,import,match,mixin,%
    new,null,object,override,package,%
    private,protected,requires,return,sealed,%
    super,this,throw,trait,true,try,%
    type,val,var,while,with,yield},
  otherkeywords={=>,<-,<\%,<:,>:,\#,@},
  sensitive=true,
  morecomment=[l]{//},
  morecomment=[n]{/*}{*/},
  morestring=[b]",
  morestring=[b]',
  morestring=[b]"""
}

%\lstset{
%    float=hbp,
%    basicstyle=\ttfamily\color{black}\small,
%    identifierstyle=\color{colIdentifier},
%    %keywordstyle=\color{colKeys},
%    %stringstyle=\color{colString},
%    %commentstyle=\color{colComments},
%    keywordstyle=\color{blue},
%    commentstyle=\color{dkgray},
%    stringstyle=\color{dkgreen},
%    columns=flexible,
%    tabsize=2,
%    frame=single,
%    extendedchars=true,
%    showspaces=false,
%    showstringspaces=false,
%    numbers=left,
%    numberstyle=\tiny\color{gray},
%    breaklines=true,
%    backgroundcolor=\color{bluegray},
%    breakautoindent=true
%}
\lstset{
  basicstyle=\ttfamily\color{black}\footnotesize,
  numbers=left,               % Ort der Zeilennummern
  numberstyle=\tiny,          % Stil der Zeilennummern
  %stepnumber=2,               % Abstand zwischen den Zeilennummern
  numbersep=5pt,              % Abstand der Nummern zum Text
  tabsize=2,                  % Groesse von Tabs
  extendedchars=true,         %
  breaklines=true,            % Zeilen werden Umgebrochen
  numberstyle=\tiny\color{dkgray},
  keywordstyle=\color{blue},
  commentstyle=\color{orange},
  stringstyle=\color{dkgreen},
  frame=tb,
  %keywordstyle=[1]\textbf,    % Stil der Keywords
  %keywordstyle=[2]\textbf,    %
  %keywordstyle=[3]\textbf,    %
  %keywordstyle=[4]\textbf,   \sqrt{\sqrt{}} %
  %stringstyle=\color{white}\ttfamily, % Farbe der String
  showspaces=false,           % Leerzeichen anzeigen ?
  showtabs=false,             % Tabs anzeigen ?
  xleftmargin=0pt,
  framexleftmargin=2pt,
  framexrightmargin=2pt,
  framexbottommargin=0pt,
  backgroundcolor=\color{lstBg},
  showstringspaces=false      % Leerzeichen in Strings anzeigen ?
}

\usepackage{listings}
\usepackage{xcolor}

\colorlet{punct}{red!60!black}
\definecolor{background}{HTML}{EEEEEE}
\definecolor{delim}{RGB}{20,105,176}
\colorlet{numb}{magenta!60!black}

\lstdefinelanguage{json}{
    basicstyle=\normalfont\ttfamily,
    numbers=left,
    numberstyle=\scriptsize,
    stepnumber=1,
    numbersep=8pt,
    showstringspaces=false,
    breaklines=true,
    frame=lines,
    backgroundcolor=\color{background},
    literate=
     *{0}{{{\color{numb}0}}}{1}
      {1}{{{\color{numb}1}}}{1}
      {2}{{{\color{numb}2}}}{1}
      {3}{{{\color{numb}3}}}{1}
      {4}{{{\color{numb}4}}}{1}
      {5}{{{\color{numb}5}}}{1}
      {6}{{{\color{numb}6}}}{1}
      {7}{{{\color{numb}7}}}{1}
      {8}{{{\color{numb}8}}}{1}
      {9}{{{\color{numb}9}}}{1}
      {:}{{{\color{punct}{:}}}}{1}
      {,}{{{\color{punct}{,}}}}{1}
      {\{}{{{\color{delim}{\{}}}}{1}
      {\}}{{{\color{delim}{\}}}}}{1}
      {[}{{{\color{delim}{[}}}}{1}
      {]}{{{\color{delim}{]}}}}{1},
}