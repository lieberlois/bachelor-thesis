% Eigene Befehle und typographische Auszeichnungen für diese

% einfaches Wechseln der Schrift, z.B.: \changefont{cmss}{sbc}{n}
\newcommand{\changefont}[3]{\fontfamily{#1} \fontseries{#2} \fontshape{#3} \selectfont}

\newcommand{\origttfamily}{}
\let\origttfamily=\ttfamily %Voheriges \ttfamily sichern
\renewcommand{\ttfamily}{\origttfamily \hyphenchar\font=`\-}

%highlighted texttt
\newcommand{\hltexttt}[1]{\texttt{\hl{#1}}}

% Abkürzungen mit korrektem Leerraum 
\newcommand{\Ua}{\mbox{U.\,a.\ }}
\newcommand{\ua}{\mbox{u.\,a.\ }}
\newcommand{\ZB}{\mbox{Z.\,B.\ }}
\newcommand{\zB}{\mbox{z.\,B.\ }}
\newcommand{\dahe}{\mbox{d.\,h.\ }}
\newcommand{\Vgl}{Vgl.\ }
\newcommand{\vgl}{vgl.\ }
\newcommand{\Bzw}{Bzw.\ }
\newcommand{\bzw}{bzw.\ }
\newcommand{\Bspw}{Bspw.\ }
\newcommand{\bspw}{bspw.\ }
\newcommand{\Evtl}{Evtl.\ }
\newcommand{\evtl}{evtl.\ }

\newcommand{\Uebs}[1]{Übs.: #1}
\newcommand{\uebs}[1]{übs.: #1}

\newcommand{\abbildung}[1]{Abbildung~\ref{fig:#1}}

\newcommand{\bs}{$\backslash$}

% erzeugt ein Listenelement mit fetter Überschrift 
\newcommand{\itemd}[2]{\item{\textbf{#1}}\\{#2}}

% einige Befehle zum Zitieren --------------------------------------------------
\newcommand{\Zitat}[2][\empty]{\ifthenelse{\equal{#1}{\empty}}{\citep{#2}}{\citep[#1]{#2}}}

% zum Ausgeben von Autoren
%\newcommand{\AutorName}[1]{\textsc{#1}}
\newcommand{\AutorName}[1]{{#1}}
\newcommand{\Autor}[1]{\AutorName{\citet*{#1}}}

% verschiedene Befehle um Wörter semantisch auszuzeichnen ----------------------
\newcommand{\Begriff}[1]{\textbf{#1}}
\newcommand{\Fachbegriff}[1]{\textit{#1}}

\newcommand{\Eingabe}[1]{\texttt{#1}}
\newcommand{\Code}[1]{\hltexttt{#1}}
\newcommand{\Datei}[1]{\texttt{#1}}

\newcommand{\Datentyp}[1]{\textsf{#1}}
\newcommand{\XMLElement}[1]{\textsf{#1}}
\newcommand{\Webservice}[1]{\textsf{#1}}


\newcommand{\footcite}[1]{\footnote{\citealp{#1}}}
\newcommand{\footuebscite}[2]{\footnote{\citealp{#1} \Uebs{#2}}}
\newcommand{\footvglcite}[1]{\footnote{\Vgl\citealp{#1}}}