% \begin{quote}
% ">Das Problem mit Zitaten aus dem Internet ist, dass man nie weiß ob sie echt sind."<
% \end{quote}

% \begin{figure}
%   \begin{center}
%     \includegraphics[width=0.6\textwidth]{./images/hochschule.jpg}
%   \end{center}
%   \vspace{-5pt}
%   \caption[Hochschule Augsburg]{Hochschule Augsburg} % Eckige Klammer (optional): Caption-Text in Abbildungsverzeichnis
%   \label{fig:hochschule}
%   \vspace{-5pt}
% \end{figure}

% Lorem ipsum dolor sit amet, consetetur sadipscing elitr, sed diam nonumy eirmod \Fachbegriff{tempor} invidunt ut labore et dolore magna aliquyam erat, sed diam voluptua. At vero Abbildung~\ref{fig:standort_rotes_tor_anfahrt_klm_bau} eos et accusam et justo duo dolores et ea rebum.

% \paragraph*{Beispiel}

% \section{Schoko 2}

% \begin{samepage}
%   Lorem ipsum dolor sit amet, consetetur sadipscing elitr, sed diam nonumy eirmod tempor invidunt ut labore et dolore magna aliquyam erat, sed diam voluptua:
%   \begin{description}
%     \item[Lorem ipsum] dolor sit amet
%     \item[consetetur] sadipscing elitr
%     \item[diam nonumy] eirmod tempor invidunt
%     \item[labore et dolore] magna aliquyam erat
%   \end{description}
% \end{samepage}

% Mehr im Kapitel~\ref{sec:ThHi:first}.

% \begin{itemize}
% \item vero eos
% \item accusam
% \item justo duo dolores
% \item ea rebum
% \end{itemize}


% Beispiel für Quelltexte (Siehe Listing~\ref{lst:HelloWorld}):

% \begin{minipage}{\textwidth}
%   \captionof{lstlisting}[Hello World]{Hello World} % Eckige Klammer (optional): Caption-Text in Listingsverzeichnis
%   \vspace{-3pt}
%   \begin{lstlisting}[language=java,label=lst:HelloWorld]
% public class HelloWorld
% {
%   public static void main(String[] args)
%   {
%     System.out.println("HelloWorld");
%   }
% }
%   \end{lstlisting}
% \end{minipage}

% Fachbegriffe werden \Fachbegriff{kursiv} formatiert.

% Klassennamen und einzeilige Codefragmente werden in \Code{Proportionalschrift}, längerer Quelltext in Form von Codeblöcken, die als Listings bezeichnet werden, dargestellt.

% Zitate und Metaphern werden in ">doppelte Anführungszeichen"< gestellt.

% Liegt eine besondere Betonung auf einem Wort, so wird dieses \textbf{fettgedruckt} dargestellt.
% Sonstige Hervorhebungen werden ebenfalls \textbf{fettgedruckt}.

% Abkürzungen werden bei erster Nennung kurz erläutert und können zudem im Abkürzungsverzeichnis auf Seite \pageref{sec:Glossar} nachgeschlagen werden.