\section*{Zusammenfassung}
\label{sec:Zusammenfassung}

Mit dem schnellen Wachstum des Internet of Things und der Vielzahl an Möglichkeiten für derartige Systeme stellt sich die Frage, welche Funktechniken, Software\-komponenten und Hardware für eigene Anwendungsfälle einerseits benötigt werden und andererseits am besten geeignet sind. Die bisher verbreitesten Funktechnologien wie Zigbee oder Bluetooth Low Energy können heute oft nicht mehr mit den aktuellen Usecases von IoT-Systemen mithalten. Die Techniken scheitern oft bereits an elementar wichtigen Anforderungen wie zum Beispiel großer Reich\-weite, hoher Skalierbarkeit oder geringem Energieverbrauch. In dieser Bachelorarbeit werden verschiedene der bisher beliebtesten Protokolle und Architekturen für IoT-Szenarien analysiert. Besonders intensiv wird das moderne Kommunikationsprotokoll \Begriff{LoRa} beziehungsweise \Begriff{LoRaWAN} behandelt. Neben technischen Details des Protokolls wird außerdem auf die Funktionsweise und die beteiligten Komponenten eines solchen Netzwerks eingegangen. Des Weiteren beschäftigt sich die Arbeit in einem praktischen Teil mit dem Aufbau zweier Prototypen, wobei der eine Prototyp größtenteils aus externen Services und der andere ausschließlich aus Open-Source-Softwarekomponenten besteht. Außerdem soll eine Einschätzung über die Praktikabilität von LoRa als Protokoll für IoT-Szenarien, im Speziellen für Smart Building Services, vorgenommen werden.

Aus der Arbeit geht das Ergebnis hervor, dass LoRa ein ideales Protokoll für viele IoT-Szenarien ist, gerade wenn große Distanzen und niedriger Energieverbrauch essentiell wichtige Anforderungen sind. Gerade für Smart Building Lösungen scheint das Protokoll besonders gut geeignet zu sein. Einer der Gründe hierfür ist, dass bereits eine Empfängerstation für große Gebäude ausreichend ist, da derartige Stationen mit einer Vielzahl von Endgeräten umgehen können und über eine große Reichweite verfügen. Limitierungen von LoRa sind hauptsächlich eine geringe Bandbreite und eine stark limitierte Anzahl an Nachrichten. Die niedrige Anzahl der sendbaren Nachrichten pro Tag stellt für die meisten IoT-Anwendungsfälle jedoch kein Problem dar. Aufgrund der niedrigen Hardwarekosten und der kostenfreien Nutzung des unlizenzierten Funkspektrums eignet sich LoRa außerdem sehr gut für private Anwendungszwecke.