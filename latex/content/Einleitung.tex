\chapter{Einleitung}
\label{cha:Einleitung}

Unter dem Internet of Things, kurz: IoT, versteht man die Verbindung und den Datenaustausch von Sensoren und Aktoren, sogenannter Things. Der Begriff wurde kurz vor der Jahrhundertwende erstmals verwendet, fand jedoch in den folgenden Jahren keine große Aufmerksamkeit. In den vergangenen Jahren hingegen wurde der Markt für IoT-Lösungen zu einem wahren Massenmarkt. Things sind heute nicht mehr nur Sensoren für industrielle Zwecke, sondern finden Anwendung im privaten Umfeld in Smart Homes oder sogar in tragbaren Geräten wie Fitness-Trackern \Zitat{Lueth.2014}. Besonders interessant sind hierbei IoT-Lösungen, deren Datenübertragung drahtlos funktioniert, da kabelgebundene Geräte für viele Vorhaben keine sinnvolle Option darstellen. Es wird davon ausgegangen, dass im Jahr 2020 über 50 Milliarden Geräte über Funktechnologien verbunden waren.

\section{Motivation}
\label{sec:Einl:Motivation}

Bisher wurden Protokolle wie Zigbee oder Bluetooth, welche durch ihre Einfachheit ideal für Smart Homes scheinen, am häufigsten für IoT-Lösungen genutzt. Derzeit geht der Trend hingegen zu sogenannten LPWAN-Technologien über, welche nicht durch hohe Bandbreite sondern durch niedrigen Energieverbrauch, niedrige Kosten und hohe Reichweite überzeugen. Zu den verbreitetsten Technologien gehören hierbei das proprietäre IoT-Netzwerk \Fachbegriff{Sigfox} und das auf dem Mobilfunknetz basierende Protokoll \Fachbegriff{NB-IoT} \Zitat{Mekki.2018}. Eines der neusten LPWAN-Protokolle ist \Fachbegriff{LoRa} und das darauf aufbauende Protokoll \Fachbegriff{LoRaWAN}, mit welchem sich diese Arbeit primär beschäftigt. LoRa soll Kommunikation über große Distanzen mit sehr geringem Energieverbrauch zu geringen Kosten ermöglichen, weshalb sich die Frage stellt, ob das Protokoll für Smart Building Services und vergleichbare Szenarien geeignet ist. 


\section{Ziel der Arbeit}
\label{sec:Einl:Ziel_der_Arbeit}

Das Ziel dieser Arbeit ist es, eine fundierte Einschätzung darüber zu geben, wie praktikabel das Protokoll LoRa für IoT-Lösungen, im Speziellen Smart Building Services, ist. Aus einem detaillierten Vergleich mit typisch verwendeten Protokollen sollen Stärken und Schwächen von LoRa für derartige Vorhaben hervorgehen. Da gerade mit LoRaWAN, einem auf LoRa aufbauenden Protokoll, diverse Systemarchitekturen möglich sind, soll außerdem durch das Erstellen von zwei Prototypen mit unterschiedlichen Techniken erforscht werden, welche Techniken für welche Szenarien am besten geeignet sind. Die Prototypen sollen jeweils ein LoRaWAN-Netzwerk und eine vollständige Datenverwaltung beinhalten. Daten sollen außerdem visualisierbar und in Echtzeit überwachbar sein. Auf Datenanomalien soll über Kommunikations\-kanäle wie E-Mail reagiert werden können. Außerdem sollen es die Prototypen erlauben, Zugriffsrechte zu verwalten, um die Prototypen auch für größere Personengruppen problemlos nutzbar zu machen. Es soll zudem einfach möglich sein, neue Geräte zum System hinzuzufügen.


\section{Aufbau der Arbeit}
\label{sec:Einl:Aufbau_der_Arbeit}

Die Arbeit beginnt damit, existierende IoT-Protokolle und die damit verbundenen Netzwerkarchitekturen zu analysieren. Hier werden die Protokolle außerdem in verschiedene Klassen eingeordnet, um später besser verstehen zu können, wofür sich das Protokoll LoRa besonders gut eignet, beziehungsweise wofür es eher nicht verwendet werden sollte. Im weiteren Verlauf wird LoRa, und das damit verbundene Protokoll LoRaWAN, ausführlich dargelegt. Nachdem kurz auf die Entstehungsgeschichte eingegangen wurde, wird anschließend die Verbindung von LoRa zu LoRaWAN geklärt. Danach werden technische Details des Protokolls beschrieben. Im Anschluss darauf folgt die Beleuchtung einer typischen LoRaWAN-Netzwerkarchitektur, sowohl von Hardware- als auch von Softwareseite. Auch auf die Sicherheit und die Verschlüs\-selung des Protokolls soll hier eingegangen werden. Der Abschluss des Kapitels behandelt das größte öffentliche LoRaWAN-Netzwerk namens \Fachbegriff{The Things Network} und das Unternehmen \Fachbegriff{The Things Industries}. Darauf folgt ein Kapitel, das sich mit dem Aufsetzen, Testen und Evaluieren zweier verschiedener Prototypen beschäftigt. In diesem Abschnitt wird intensiv auf die verwendeten Technologien und deren Vor- und Nachteile eingegangen. Die beschriebenen Prototypen sind hierbei jeweils eine Version, die größtenteils fertige Software-as-a-Service Deployments nutzt, sowie eine Version, die vollständig auf Open-Source-Produkte aufbaut.
