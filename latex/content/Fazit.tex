\chapter{Fazit}
\label{cha:fazit}

In dieser Bachelorarbeit wurde das Funkprotokoll LoRa ausführlich analysiert und anhand von praxisorientierten Tests in Form von zwei verschiedenen Prototypen auf die Eignung für Smart Building IoT-Services geprüft.

\section{LoRa im Vergleich}

Um das Protokoll LoRa besser einordnen zu können, wurden zu Beginn typisch verwendete IoT-Protokolle betrachtet. Es fiel schnell auf, dass die bisher meist verwendeten Wireless Private Area Network (WPAN) Techniken wie Zigbee und Bluetooth Low Energy zwar weiterhin sehr beliebt bleiben, jedoch große Schwächen für IoT-Lösungen aufzeigen. Eine dieser Schwächen ist, dass die Protokolle meist nicht energieeffizient genug arbeiten, um Endgeräte über größere Zeiträume mit Batterien betreiben zu können. Außerdem sind die maximalen Sendereichweiten von etwa 100 Metern für viele Szenarien nicht ausreichend, besonders da die Reichweite durch Hindernisse wie beispielsweise Wände stark verringert wird. Darüber hinaus wurde festgestellt, dass WPAN-Techniken für sehr hohe Geräteanzahlen eher ungeeignet sind. Es sind jedoch auch klare Vorteile der betrachteten Techniken aufgefallen. Protokolle wie Zigbee und BLE überzeugen durch Einfachheit sowohl bei der Nutzung als auch in den Netzwerkstrukturen selbst. Außerdem senden die Protokolle mit verhältnismäßig hohen Bandbreiten und geringen Latenzen, weswegen die Protokolle auch für die Datenübertragung in Echtzeit geeignet sind.

Nach den WPAN-Techniken wurden in der Arbeit Low Power Wide Area Network (LPWAN) Techniken behandelt. Techniken dieser Klasse haben in den vergangen Jahren gerade im IoT-Bereich immens an Beliebtheit gewonnen und lösen viele Probleme, die traditionelle IoT-Protokolle aufweisen. Mit LPWAN-Techniken ist es möglich, Daten über große Distanzen mit sehr geringem Energieaufwand zu versenden. Netzwerke sind sehr flexibel und skalierbar, wobei bereits einige Netzwerke große Teile der Erde abdecken. Die beiden Protokolle Sigfox und NB-IoT gehören zur Klasse der LPWAN-Techniken und wurden in der Arbeit präziser betrachtet. So baut NB-IoT auf das bereits existierende Mobilfunknetz auf und weist dadurch eine große Netzwerkabdeckung vor. Der größte Vorteil des Protokolls ist die sehr reife Infrastruktur, die zum Mobilfunk gehört. So ist beispielsweise Datensicherheit oder Skalierbarkeit des Netzes bereits vorhanden. Auch die hohe Energieeffizienz spricht für NB-IoT. Da das Protokoll im lizenzierten Spektrum operiert, ist die Nutzung des Netzes jedoch mit vergleichsweise hohen Kosten verbunden. Außerdem ist es Nutzern nicht möglich, eigene Netze zu erstellen, wodurch eine große Abhängigkeit entsteht. Sigfox bietet ebenfalls ein bereits existierendes Netzwerk an und verfolgt das Ziel, ein globales Netzwerk aufzubauen. Zur Nutzung des Sigfox-Netzwerks gehört die Sigfox-Cloud, in die Daten von Endgeräten über Empfangsstationen weitergeleitet und gespeichert werden. Sigfox überzeugt durch ein Gesamtpaket von preiswerten Geräten, einer großen Netzwerkabdeckung und einem kompletten Device-To-Cloud Datenversand. Jedoch ist auch die Nutzung von Sigfox mit Problemen verbunden. Wie auch bei NB-IoT ist es nicht möglich, eigene Netzwerke zu erstellen, was gerade dann problematisch ist, wenn die Zielregion nicht durch das Sigfoxnetzwerk abgedeckt ist. Da Sigfox im frei nutzbaren und unlizenzierten Spektrum arbeitet, sind zwar die Kosten gering, jedoch wird das Sendeverhalten im Netzwerk stark eingeschränkt. Mit Sigfox kann ein Gerät lediglich 140 Nachrichten am Tag senden, wodurch Szenarien, in denen das Senden von Echtzeitdaten oder eine zuverlässige Berichterstattung erforderlich sind, mit Sigfox nicht realisiert werden können.

LoRa und das damit verbundene Protokoll zählt zu den LPWAN-Techniken und ist von den erwähnten Protokollen am ehesten mit Sigfox in Verbindung zu bringen. So ist auch LoRa ein Protokoll, mit welchem Daten über große Distanzen mit geringem Energieverbrauch gesendet werden können. Das Protokoll ist außerdem sehr robust gegen Interferenzen und weitere typische Funkprobleme. Wie auch Sigfox operiert LoRa im unlizenzierten Spektrum und kann somit frei genutzt werden. Normalerweise muss man sich bei der Nutzung des Spektrums lediglich an den sogenannten Duty Cycle halten. Dieser reguliert die erlaubte Sendezeit auf einen festgelegten Prozentsatz eines Zeitfensters. Dieser Wert variiert jedoch je nach Region, wobei der Duty Cycle für LoRa in Deutschland bei 1\%, also beispielsweise 36 Sekunden pro Stunde liegt. Bei externen Netzanbietern gibt es jedoch meist noch weitere Regulierungen. Für typische IoT-Szenarien, in denen wenige und vor allem kleine Datenpakete versendet werden, ist dies aber in der Regel kein Problem.\\ 
Während das proprietäre Protokoll LoRa lediglich die erste Schicht des TCP/IP-Schichtenmodells abdeckt, ist das Open-Source-Protokoll LoRaWAN für die darüber liegenden Schichten zuständig und definiert außerdem die Struktur des Netzwerks. Typische LoRaWAN-Netzwerke sind in die verschiedenen Komponenten Packet Forwarder, Gateway Bridge, Join Server, Network Server und Application Server aufgeteilt, wobei der Network Server den Kern des Netzwerks und somit den Stern der Sterntopologie darstellt. Zwar existieren hierfür viele externe Netzwerkanbieter, es ist jedoch, anders als bei Sigfox und NB-IoT, möglich, eigene Netzwerke aufzubauen. Wie auch beim Protokoll Sigfox erreicht man durch einen LoRaWAN-Stack eine Device-To-Cloud-Connectivity. Auch Sicherheit ist bei LoRaWAN ein großes Thema. Die Sicherheit von LoRaWAN basiert auf einer AES-Verschlüsselung und einem System aus verschiedenen Keys, sodass selbst Komponenten im Netzwerk nur den für sie relevanten Teil einer Nachricht sehen können. Als Abschluss des theoretischen Hintergrunds wurden neben einer kurzen Beschreibung des \mbox{Messaging} Protokolls MQTT das The Things Network und das Unternehmen The Things Industries vorgestellt. Beim The Things Network handelt es sich um ein globales, öffentliches LoRaWAN-Netzwerk, wobei jeder freien Zugriff auf alle teilnehmenden Empfangsstationen erhält und auch eigene Stationen ans Netzwerk anbinden kann. The Things Industries bietet neben Consulting und fertigen LoRaWAN-Lösungen für das The Things Network außerdem vorkonfigurierte LoRaWAN-Geräte an, um Kunden den Einstieg zu erleichtern.

\section{Netzwerkarchitekturen für IoT mit LoRa}

Im weiteren Verlauf der Arbeit wurden zwei verschiedene LoRaWAN-Netzwerk-Prototypen behandelt. Mithilfe dieser Prototypen sollte erforscht werden, welche der unzähligen Auswahlmöglichkeiten an Softwarekomponenten für welche Szenarien geeignet sind. Als Endgeräte wurden hierfür jeweils ein Temperatur- und Luftfeuchtigkeitssensor und ein Türsensor verwendet. Außerdem wurde dargestellt, warum für die Speicherung von IoT-Daten Time-Series-Datenbanken mit am besten geeignet sind. 

Beim ersten Prototypen war die Nutzung möglichst vieler, bereits existierender, Hardware- und Softwarekomponenten das Ziel. Die Erwartungen an den Prototypen waren hierbei vor allem ein schnelles und einfaches Setup des Netzwerks mit wenig Konfigurationsaufwand, gute Skalierbarkeit und die Möglichkeit, eingehende Daten zu visualisieren und auf diese in Echtzeit zu reagieren. Die Kernkomponenten des Prototypen waren hierbei im LoRaWAN-Teil das The Things Network und im Datenverwaltungsteil Microsofts Azure IoT Central. Wie bereits erwartet, war das Hinzufügen der Endgeräte ins LoRaWAN-Netzwerk sehr einfach, da keine Server oder gar Services aufgesetzt werden mussten. Da der Arbeitsplatz, an dem der Prototyp aufgebaut wurde, bereits durch ein fremdes LoRaWAN-Gateway abgedeckt war, wäre beim Aufbau kein eigenes Gateway nötig gewesen. Hierdurch wären Hardwarekosten des Systems deutlich geringer ausgefallen. Da jedoch bei der Nutzung keine Garantie bestand, dass fremde Gateways konstant erreichbar sein würden, wurde sicherheitshalber trotzdem ein Gateway von The Things Industries in Betrieb genommen.\\
Auch das Aufsetzen einer Azure IoT Central Anwendung selbst verlief problemlos. Besonders positiv fiel hierbei die Möglichkeit auf, die Datenstruktur der verschiedenen Endgeräte in Form sogenannter Device Templates zu definieren, um ein semantisches Mapping eingehender Daten zu erhalten, was die Datennutzung sehr erleichtert. Als weitere Stärke fiel die Integration in die Microsoft-Infrastruktur auf. So können in einer Azure IoT Central Anwendung beispielsweise in kürzester Zeit \mbox{E-Mail}-Alerts beim Eintritt festgelegter Regeln konfiguriert werden, ohne dafür einen E-Mail-Server hinterlegen zu müssen. Auch die Integration der Nutzerverwaltung von Microsoft funktionierte hier tadellos. Weiter konnten nun beliebige \mbox{Microsoft} Softwarekomponenten wie beispielsweise eine Time-Series-Datenbank zur permanenten Speicherung eingehender Daten ins Netzwerk hinzugefügt werden.\\
Besonders schwierig war beim Aufbau des Prototypen die Verbindung zwischen dem The Things Network und der Azure IoT Central Anwendung. Hierfür wurde eine Azure Function, eine Art von HTTP-Endpoint, genutzt und in Microsoft Azure bereitgestellt. Per HTTP-Integration konnten Daten aus dem The Things Network nun über den HTTP-Endpoint in die IoT Central Anwendung weitergeleitet werden. Dies ist als Schwäche dieses Prototypen anzusehen, da dies unverhältnismäßig aufwändig für eine Verbindung des LoRaWAN-Netzes mit der Datenverwaltung ist. Außerdem sind grundlegende Programmierkenntnisse sowie fortgeschrittene Kenntnisse in einer Cloud-Plattform wie beispielsweise Microsoft Azure für diese Verbindung erforderlich. Allgemein überzeugte der Prototyp vor allem durch das einfache Setup der beiden Hauptkomponenten und die Nutzung des öffentlichen LoRaWAN-Netzwerks.

Beim Aufbau des zweiten Prototypen wurden ausschließlich Komponenten mit einer Open-Source-Lizenz genutzt und selbst in einer virtuellen Maschine in Microsoft Azure mithilfe von Docker aufgesetzt. Zu erwähnen gilt hierbei, dass dieser Prototyp auch vollständig offline bereitstellbar und mit allen Funktionen nutzbar ist. Die größte Erwartung an den Prototypen war hierbei ein großer Aufwand beim Aufsetzen der Softwarekomponenten und bei der Serverkonfiguration. Da das Unternehmen The Things Industries mit der einfachen Inbetriebnahme ihrer Geräte wirbt, wurde außerdem damit gerechnet, dass das Setup eines eigenen LoRaWAN-Netzwerks aufwendig sein würde. Auch waren Erwartungen an das System ein vollständiger Einblick in Protokolle aller genutzten Services und eine höhere Anzahl an sendbaren Nachrichten, da anders als beim The Things Network in einem eigenen LoRaWAN-Netzwerk neben dem Duty Cycle keine weiteren Regulierungen existieren.
Als LoRaWAN-Stack wurde für den zweiten Prototypen der Open-Source ChirpStack verwendet, welcher dem The Things Network in der Funktionsweise sehr stark ähnelt. Das Setup des LoRaWAN-Stacks war erstaunlich einfach, da alle Komponenten in Docker Containern bereitgestellt werden können und sogar fertige Docker-Compose-Konfigurationsdateien existieren, mit denen das bereits vorkonfigurierte System in kürzester Zeit in Betrieb genommen werden kann. Das Hinzufügen des LoRaWAN-Gateways war ebenfalls einfacher als erwartet. Hierzu musste lediglich die IP-Adresse und der Port der Gateway Bridge, einer der Komponenten des ChirpStack, in der Software des Gateways hinterlegt werden. Nun konnte das Gateway in der Konfigurationssoftware des ChirpStacks ins System aufgenommen werden. Anders als beim ersten Prototypen war dieses Netzwerk nun vollkommen privat und hatte neben den eigenen Gateways keine weitere Netzwerkabdeckung. Das Sendeverhalten war jedoch lediglich durch den Duty Cycle, nicht aber wie beim The Things Network durch eine Fair Use Policy, reguliert. Durch eine durchdachte Nutzer- und Rechteverwaltung konnten Nutzern Zugriffsrechte auf die einzelnen Komponenten des Netzwerks wie Gateways oder Endgeräte erteilt werden.

Zur Datenverwaltung und -speicherung wurden als Software\-komponenten ein selbst entwickelter MQTT-Consumer zum Auslesen der Daten aus dem ChirpStack, eine InfluxDB zur permantenten Datenspeicherung und Grafana zur Datenvisualisierung genutzt. Auch diese Komponenten wurden in Docker Containern bereitgestellt. Auch die Konfiguration der Komponenten stellte sich hier als überraschend einfach heraus. Im Vergleich zum ersten Prototypen waren für diesen Prototypen deutlich mehr Kenntnisse in der Softwareentwicklung, Azure und allgemeiner Serververwaltung nötig. Auch wichtige Elemente wie eine Backupstrategie oder Kommunikationskanäle für Alerts müssten mit diesem Prototypen selbst konfiguriert werden, während die externen Services beim ersten Prototypen viele dieser Arbeiten abnehmen. Beim zweiten Prototypen fielen allgemein die unzähligen Möglichkeiten zur Datenvisualisierung in Grafana, die vollständige Kontrolle über alle Softwarekomponenten und die Flexibilität durch die verteilte Netzwerkarchitektur sehr positiv auf.

\section{Zusammenfassung}

Zusammenfassend lässt sich sagen, dass das Protokoll LoRa für Zwecke wie Smart Buildings aber auch für Smart City Szenarien, die sich über größere Gebiete wie Städte oder Landkreise erstrecken, sehr gut eignet. Das Protokoll liefert verbunden mit LoRaWAN eine für IoT-Lösungen hervorragende Balance aus Sendereichweite, Bandbreite, Energieverbrauch und Kosten und liegt in einer Nische zwischen dem sehr ähnlichen, aber stark regulierten, Protokoll Sigfox und der auf den Mobilfunk basierenden Technik NB-IoT. LoRa und LoRaWAN überzeugen verglichen mit diesen Techniken durch die Flexibilität beim Netzwerkaufbau und einem nur wenig beschränkten Sendeverhalten der freien Nutzung des Funkspektrums. Während das Protokoll für typische IoT-Lösungen sehr gut geeignet ist, sollte in Usecases, in denen Echtzeitdaten oder eine sehr hohe Bandbreite erforderlich sind, eher zu klassischen PWAN-Techniken wie BLE oder Zigbee gegriffen werden. Sind die große Reichweite, aber auch eine hohe Nachrichtenanzahl unabdingbar, sollte eine mobilfunkbasierte Technik wie NB-IoT genutzt werden, wobei hier die Geräte- und Netzwerkkosten deutlich höher sind.

Die Frage, welche Netzwerkarchitektur am sinnvollsten genutzt werden sollte, hängt von den Anforderungen ans System ab. Das The Things Network als LoRaWAN-Netzwerk erlaubt den einfachsten Einstieg in das Thema LoRa. Das Netzwerk kann genutzt werden, ohne selbst Server oder Services aufsetzen zu müssen. Die Nutzung des öffentlichen Netzwerks spart nicht nur Kosten, sondern bietet Nutzern zukünftig die Möglichkeit, global LoRaWAN-Signale zu empfangen. Für große IoT-Netzwerke mit einem hohen Maß an Nachrichten kommt das The Things Network jedoch schnell an seine Grenzen. Zwar bietet The Things Industries eigene, maßgeschneiderte Deployments für große Netzwerke an, jedoch werden diese schnell sehr kostspielig. Die Prototypen erwecken daher den Eindruck, dass sich die Bereitstellung eines eigenen LoRaWAN-Stacks wie beispielsweise dem ChirpStack für große oder gar kommer\-zielle Netze besser eignet. Durch das Aufsetzen eines privaten Netzwerks fallen viele Regulierungen wie beispielsweise die Fair Access Policy im The Things Network weg und erlauben einen flexibleren Datenaustausch mit Endgeräten. Außerdem erhält der Nutzer volle Kontrolle über alle Netzwerkkomponenten und kann so Dinge wie die Verwaltung von Logs, Datenspeicherung oder verteilte Deployments selbst steuern. Außerdem können private Netzwerke offline oder in einem Intranet erstellt werden, während öffentliche Netzwerke stets einen Internetzugang benötigen. Die netzwerkinterne Kommunikation läuft aufgrund der Modularität der Softwarekomponenten typischerweise zum Großteil über Messaging Protokolle wie MQTT oder Remote Procedure Call Protokollen wie gRPC. Bei beiden LoRaWAN-Stacks gilt es allgemein nicht zu vergessen, dass beide Stacks noch täglich verbessert werden, und somit nicht anhand einzelner, fehlender Features wie mehreren Decoder-Funktionen pro Applikation im The Things Network bewertet werden sollten. 

Auf der Seite der Datenverwaltung und -visualisierung können ebenfalls beide Versionen sinnvoll eingesetzt werden. Azure IoT Central entpuppte sich besonders dann als sinnvoll, wenn das System von einer Integration in die Microsoft Infrastruktur profitiert. Die existierende Microsoft Nutzerverwaltung, das einfache Senden von Nachrichten über verschiedene Kommunikationskanäle wie E-Mail und das Hinzufügen unzähliger Microsoft-Services sprechen sehr für die IoT-Plattform. Jedoch ist mit der Nutzung von Azure IoT Central das Aufsetzen einer IoT Device Bridge verbunden, welche Daten zwischen dem LoRaWAN-Netzwerk und der IoT Central Anwendung austauscht. Auch die stark eingeschränkten Möglichkeiten zur Visualisierung eingehender Daten sind ein Nachteil von Azure IoT Central Anwendungen. Mit dem alternativen Open-Source-Stack bestehend aus einer InfluxDB als Datenbank und Grafana zur Datenvisualisierung erhält man eine einfach nutzbare Plattform zur Datenverwaltung. Besonders überzeugend ist hierbei Grafana mit seinen unzähligen Möglichkeiten zur Darstellung von Daten. Da jedoch in diesem Stack keine Anbindung an eine existierende Infrastruktur besteht, müssen Dinge wie eine Nutzerverwaltung oder Kommunikationskanäle für Alerts selbst aufgebaut werden. Stellt dies für den Nutzer kein Problem dar, so kann man anhand der in der Arbeit erstellten Prototypen schlussfolgern, dass der Open-Source-Stack auf Datenverwaltungsebene für die meisten Szenarien besser geeignet ist.

\section{Ausblick}

Da in dieser Arbeit allgemein nur wenig auf die Hardwareseite von LoRa eingegangen wurde, bietet sich hier ein großes Potenzial für weitere Forschungen. Neben der genauen Funktionsweise des Datentransports über LoRa oder des Aufbaus von LoRa-Geräten könnte es außerdem profitabel sein, Strukturen zur zuverlässigen Anbindung von IoT-Lösungen mit LoRa an ansteuerbare Hardware wie Klimaanlagen oder automatisierte Bewässerungsanlagen zu analysieren.